% Article
\documentclass[]{article}
% Input encoding is UTF-8
\usepackage[utf8]{inputenc}
% Package for hypertext
\usepackage{hyperref}
% Package for graphics
\usepackage{graphicx}

% Reference: https://www.overleaf.com/learn/latex/Learn_LaTeX_in_30_minutes

\title{A Hello, World! for \LaTeX{} (and reference)}

\author{
    Avneesh Mishra
    \thanks{This is author 1 for this document} \\
    \texttt{123avneesh@gmail.com}
    \and
    Lorem ipsum 
    \thanks{This is author 2 for this document} \\
    \texttt{consectetur@dolor.sit}
}

\date{September 18, 2021}

% Main document beginning
\begin{document}

\maketitle

\begin{abstract}
    
    A basic document highlighting key \LaTeX{} features and walking through setup using VSCode (because it's simply better). The tested OS is \texttt{Ubuntu 20.04}, and \texttt{Windows}.

\end{abstract}

Most of this document is created from the following sources (can be used as reference)

\begin{enumerate}
    \item \href{https://www.overleaf.com/learn/latex/Learn_LaTeX_in_30_minutes}{Overleaf: Learn LaTeX in 30 minutes}
\end{enumerate}

\tableofcontents

\section{Installing everything}

\subsection{Prerequisites}

You must have the following installed

\begin{itemize}
    \item VSCode: The best editor for typing
    \item Ubuntu 20.04: Ubuntu (or any Debian based distro) makes it easy to install things (only commands). 
    \item This has also been tested on Windows 10, but with a few modifications.
\end{itemize}

\subsection[Setup]{System setup}

\subsubsection*{TeX Live on Ubuntu}

TeX Live is a software suite that bundles all \LaTeX{} related work in a single package that can be installed. On \emph{Ubuntu} (or any \emph{Debian} distribution), you can check running \texttt{sudo apt install texlive-full}, which will install everything.
If this doesn't work, you can try installing directly through \href{https://ctan.org/pkg/texlive}{CTAN} (Comprehensive \TeX{} Archive Network) and get installation instructions from \href{https://www.tug.org/texlive/}{here}.

\noindent TEX Live is simply the best distribution that packages a lot of important things for \LaTeX{}. You can find others from \href{https://www.latex-project.org/get/}{here}.

\noindent Another helpful package used for \emph{linting} \LaTeX{} is \href{https://www.nongnu.org/chktex/}{chktex}. This comes bundled in \emph{TeX Live}.

\subsubsection*{MiKTeX on Windows 10}

On Windows, it's better to use \href{https://miktex.org/}{MiKTeX} to handle \LaTeX{} packages. Install it from \href{https://miktex.org/download}{here}. After installing it for the local user, add the path to the \texttt{PATH} environment variable.

\noindent Post installing MiKTeX, since it is a minimal distribution, you have to install a few packages. First, install perl through \href{https://strawberryperl.com/}{strawberryperl for Windows}. Second, install \texttt{latexmk} on MiKTeX (through \texttt{MiKTeX Console}). This will allow generation / making of \LaTeX{} documents. You can also check if \texttt{ChkTeX} is installed (through powershell).

\subsubsection*{VSCode}

We'll be using the \href{https://marketplace.visualstudio.com/items?itemName=James-Yu.latex-workshop}{LaTeX Workshop} VSCode extension to use \LaTeX{} with VSCode. You can also find this extension on \href{https://github.com/James-Yu/LaTeX-Workshop}{GitHub}.

\subsection{Alternatives}

Instead of all this, some alternatives are presented here

\begin{itemize}
    \item Use \href{https://www.overleaf.com/}{overleaf} to store everything online and not bother installing a thing on local system
\end{itemize}

\section{Basic \LaTeX{}}

% This is a comment, it's not rendered

In this section, basic \LaTeX{} syntax is explored. Here is some \textbf{bold}, \textit{italics} and \underline{underline} text. However, it's best to use \emph{emphasis} to highlight text (adapted to context). Make sure that the preamble is properly described before the \verb|\begin{document}| line (following which, the main document begins).

\subsection{Some Graphics}

To include external graphics, use the \texttt{graphicx} package.

\end{document}

